\PaperTitle{“基于共轭梯度技术的低复杂度鲁棒波束形成”复
	现报告} % Article title


\Authors{罗雪琴\textsuperscript{1},袁丹\textsuperscript{1}} % Authors

\affiliation{\quad\textsuperscript{1}\textit{西北工业大学}}

\Abstract{\phantom{田田}
	本文对基于共轭梯度法和最坏情况优化准则的低复杂度鲁棒波束形成算法进行复现,复现的主要
	算法为鲁棒约束最小方差修正共轭梯度(Robust-CMV-MCG)和鲁棒约束恒模修正共轭梯度(Robust-
	CCM-MCG),这些算法采用基于低复杂度共轭梯度算法的联合优化策略,我们将这两种算法和传统的
	加载样本矩阵求逆方法(Loaded-SMI) 方法以及基于最坏情况优化准则的约束最小方差(WC-CMV)方法进
	行对比。同时给出同条件下理论最大信干噪比作为参考。经我们的仿真和讨论发现,Robust-CMV-MCG
	和Robust-CCM-MCG在特定的情况下性能比现有鲁棒算法相当或更好,而复杂度却降低了一个数量级。
	我们小组分工明确,认真完成了复现工作,具体分工在引言部分给出,此外本文所提主要文献,代码以及参考资料在附录中给出。}


\Keywords{\phantom{田田} 鲁棒自适应波束形成\quad阵列转向\quad矢量失配\quad低复杂度算法\quad共轭梯度法}

